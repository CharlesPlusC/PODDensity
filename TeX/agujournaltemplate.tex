%%%%%%%%%%%%%%%%%%%%%%%%%%%%%%%%%%%%%%%%%%%%%%%%%%%%%%%%%%%%%%%%%%%%%%%%%%%%
% AGUJournalTemplate.tex: this template file is for articles formatted with LaTeX
%
% This file includes commands and instructions
% given in the order necessary to produce a final output that will
% satisfy AGU requirements, including customized APA reference formatting.
%
% You may copy this file and give it your
% article name, and enter your text.
%
% guidelines and troubleshooting are here: 

%% To submit your paper:
\documentclass[draft]{agujournal2019}
\usepackage{url} %this package should fix any errors with URLs in refs.
\usepackage{amsmath}
\usepackage{booktabs}
\usepackage{float}
\usepackage{lineno}
\usepackage[inline]{trackchanges} %for better track changes. finalnew option will compile document with changes incorporated.
\usepackage{soul}
\linenumbers
%%%%%%%
% As of 2018 we recommend use of the TrackChanges package to mark revisions.
% The trackchanges package adds five new LaTeX commands:
%
%  \note[editor]{The note}
%  \annote[editor]{Text to annotate}{The note}
%  \add[editor]{Text to add}
%  \remove[editor]{Text to remove}
%  \change[editor]{Text to remove}{Text to add}
%
% complete documentation is here: http://trackchanges.sourceforge.net/
%%%%%%%

\draftfalse

%% Enter journal name below.
%% Choose from this list of Journals:
%
% JGR: Atmospheres
% JGR: Biogeosciences
% JGR: Earth Surface
% JGR: Oceans
% JGR: Planets
% JGR: Solid Earth
% JGR: Space Physics
% Global Biogeochemical Cycles
% Geophysical Research Letters
% Paleoceanography and Paleoclimatology
% Radio Science
% Reviews of Geophysics
% Tectonics
% Space Weather
% Water Resources Research
% Geochemistry, Geophysics, Geosystems
% Journal of Advances in Modeling Earth Systems (JAMES)
% Earth's Future
% Earth and Space Science
% Geohealth
%
% ie, \journalname{Water Resources Research}

\journalname{Earth and Space Science}


\begin{document}

%%%%%%%%%%%%%%%%%%%%%%%%%%%%%%%%%%%%%%%%%%%%%%%
%  TITLE
%
% (A title should be specific, informative, and brief. Use
% abbreviations only if they are defined in the abstract. Titles that
% start with general keywords then specific terms are optimized in
% searches)
%
%%%%%%%%%%%%%%%%%%%%%%%%%%%%%%%%%%%%%%%%%%%%%%%

% Example: \title{This is a test title}

\title{Empirical Assessment of Storm-time Thermospheric Density Inversion Methods from LEO POD Data}

%%%%%%%%%%%%%%%%%%%%%%%%%%%%%%%%%%%%%%%%%%%%%%%
%
%  AUTHORS AND AFFILIATIONS
%
%%%%%%%%%%%%%%%%%%%%%%%%%%%%%%%%%%%%%%%%%%%%%%%

% Authors are individuals who have significantly contributed to the
% research and preparation of the article. Group authors are allowed, if
% each author in the group is separately identified in an appendix.)

% List authors by first name or initial followed by last name and
% separated by commas. Use \affil{} to number affiliations, and
% \thanks{} for author notes.
% Additional author notes should be indicated with \thanks{} (for
% example, for current addresses).

% Example: \authors{A. B. Author\affil{1}\thanks{Current address, Antartica}, B. C. Author\affil{2,3}, and D. E.
% Author\affil{3,4}\thanks{Also funded by Monsanto.}}

\authors{Charles Constant \affil{1}, Indigo Brownhall \affil{1}, Anasuya Aruliah\affil{1}, Marek Ziebart\affil{1}, Santosh Bhattarai\affil{1}}
% Author\affil{3,4}\thanks{Also funded by Monsanto.}}

\affiliation{1}{University College London}
% \affiliation{2}{Second Affiliation}
% \affiliation{3}{Third Affiliation}
% \affiliation{4}{Fourth Affiliation}

% \affiliation{1}{Gower Street, WC1E 6BT, London}
%(repeat as many times as is necessary)


% Corresponding author mailing address and e-mail address:

% (include name and email addresses of the corresponding author.  More
% than one corresponding author is allowed in this LaTeX file and for
% publication; but only one corresponding author is allowed in our
% editorial system.)

% Example: \correspondingauthor{First and Last Name}{email@address.edu}

\correspondingauthor{Charles Constant}{zcesccc@ucl.ac.uk}


%%%%%%%%%%%%%%%%%%%%%%%%%%%%%%%%%%%%%%%%%%%%%%%
% KEY POINTS
%%%%%%%%%%%%%%%%%%%%%%%%%%%%%%%%%%%%%%%%%%%%%%%
%  List up to three key points (at least one is required)
%  Key Points summarize the main points and conclusions of the article
%  Each must be 140 characters or fewer with no special characters or punctuation and must be complete sentences

% Example:
% \begin{keypoints}
% \item	List up to three key points (at least one is required)
% \item	Key Points summarize the main points and conclusions of the article
% \item	Each must be 140 characters or fewer with no special characters or punctuation and must be complete sentences
% \end{keypoints}

\begin{keypoints}
\item POD-accelerometry achieves the highest accuracy and precision vs. accelerometer-derived densities across 1887 storm-time orbits.
\item EDR performance degrades as the drag signal weakens, consistent with theoretical expectations tested here.  
\item Results support using cooperative POD streams to improve storm-time density nowcasts in operational models.
\end{keypoints}

%%%%%%%%%%%%%%%%%%%%%%%%%%%%%%%%%%%%%%%%%%%%%%%
%
%  ABSTRACT and PLAIN LANGUAGE SUMMARY
%
% A good Abstract will begin with a short description of the problem
% being addressed, briefly describe the new data or analyses, then
% briefly states the main conclusion(s) and how they are supported and
% uncertainties.

% The Plain Language Summary should be written for a broad audience,
% including journalists and the science-interested public, that will not have 
% a background in your field.
%
% A Plain Language Summary is required in GRL, JGR: Planets, JGR: Biogeosciences,
% JGR: Oceans, G-Cubed, Reviews of Geophysics, and JAMES.
% see http://sharingscience.agu.org/creating-plain-language-summary/)
%
%%%%%%%%%%%%%%%%%%%%%%%%%%%%%%%%%%%%%%%%%%%%%%%

%% \begin{abstract} starts the second page

\begin{abstract}
Thermospheric mass density represents one of the largest sources of operational uncertainty for spacecraft operating in Low Earth Orbit (LEO). This uncertainty is particularly pronounced during geomagnetic storms. State-of-the-art thermospheric models such as the High Accuracy Satellite Drag Model rely on assimilative processes by which a base model is updated using observations of thermospheric density derived from uncooperative tracking data. The burgeoning spacecraft population in LEO presents a dataset of opportunity that could provide higher-fidelity, lower-latency observations than existing uncooperative methods. By using the Precise Orbit Determination (POD) data streams generated by these satellites, one can estimate the density along their paths. We benchmark the two main existing techniques used to estimate densities from cooperative data against accelerometer-derived densities: Energy Dissipation Rate (EDR) and POD accelerometry. To contextualise the performance of these methods relative to the kinds of observations currently assimilated into operational models, we also compare them against a Two-Line Element-based method. All methods are compared over 1887 storm-time orbits across two satellites: CHAMP and GRACE-FO-A. Relative to accelerometer-derived densities, POD accelerometry achieves the strongest correlation (r² = 0.96 for GRACE-FO-A, 0.94 for CHAMP) and the lowest root-mean-square error (5.8 \%), halving the error of EDR (10.98 \%) and the uncooperative approach (13.52 \%). Finally, our analysis empirically corroborates predictions made in the literature as to the performance of these methods across varying storm-time drag environments. The application of POD-accelerometry to the growing number of satellites stands to enhance the performance of next-generation assimilative models.
\end{abstract}

\section*{Plain Language Summary}
During geomagnetic storms, Earth’s upper atmosphere ‘puffs up,’ increasing drag on satellites and making orbit predictions less reliable. We compared two ways to estimate air density from satellite tracking data that are routinely available: a method that uses changes in a satellite’s energy from orbit to orbit, and a method that derives acceleration directly from precise positions and velocities. Using independent accelerometer measurements from two science missions as the reference, we show that the acceleration-based method reproduces storm-time density changes most accurately and consistently. These findings suggest that sharing precise tracking data from today’s large satellite fleets could lead to faster, more reliable space-weather updates that improve collision avoidance and mission planning.

\section{Introduction}

The recent proliferation of satellites in Low Earth Orbit (LEO), combined with intensifying solar activity associated with the current peak of Solar Cycle 25, is heightening the operational risks posed by mischaracterisation of thermospheric behaviour. Of particular concern is the inability of current thermospheric models to accurately resolve density variations during geomagnetic storms, which remain a major source of operational uncertainty \cite{Mehta2017AAtmosphere}. This inadequacy leads to the degradation of orbital products used in space traffic management (e.g., probability of collision estimates, predicted ephemerides, state uncertainties) \cite{Bussy-Virat2018} and can occasionally culminate in uncontrolled spacecraft re-entries \cite{Fang2022Space2022}.

At present, ``the most accurate thermospheric density nowcasting capability available'' \cite{Berger2023TheOperations} is provided by the High Accuracy Satellite Drag Model (HASDM) \cite{Storz2002HASDMRates, Picone2005ThermosphericSets}, a data-assimilative model proprietary to the U.S. Government \cite{Mutschler2023AOperations}. HASDM uses a variant of the Jacchia-Bowman 2008 atmospheric density model as its base model, and updates this periodically through an assimilative scheme. In \citeA{Storz2002HASDMRates}, the observations used to drive this scheme are so-called Energy Dissipation Rates (EDR), derived from 80+ uncooperatively-tracked objects in eccentric orbits \cite{Mutschler2023AOperations}. Note that while an operational version of HASDM is not publicly available, a limited database for the year 2019 is accessible at \url{https://spacewx.com/hasdm/}.

The strong performance of HASDM and the absence of any operationally available alternative has motivated the development of new assimilative frameworks \cite{Sutton2018AThermosphere, Gondelach2020, Elvidge2019UsingModelling}. These models aim to leverage near-real-time observations of the thermosphere to improve upon existing empirical models. They assimilate a broad range of data sources beyond radar-tracking data, including incoherent scatter radar measurements, Fabry-Perot interferometer readings, accelerometer-derived densities (i.e. from CHAMP, GRACE, Swarm, GOCE) and, increasingly, Precise Orbit Determination (POD) data. Whilst work towards operationalisation of some of these is in progress \cite{Brown2024UsingSolutions}, many remain experimental in scope. Meanwhile, the margin for error in day-to-day orbit prediction is dropping with each new satellite launch, especially for conjunction analysis, making thermospheric density estimation an increasingly critical bottleneck for the space operations community \cite{Fang2022Space2022}.

The bulk of non-assimilative, operationally available empirical models such as Jacchia-Bowman 2008 (JB2008) \cite{Bowman2008AIndices}, Naval Research Laboratory Mass Spectrometer and Incoherent Scatter Radar Extended Model (NRLMSISE-00) \cite{Picone2002NRLMSISE-00Issues}, and the Drag Temperature Model 2000 (DTM2000) \cite{Bruinsma2003TheProperties}, are built in part from parameterizations fitted to historical data \cite{Gondelach2020, Foster2016OrbitConstellations, Fang2022Space2022}. These empirical models are computationally efficient and widely used, with performance well characterized by the community. However, their resolution of thermospheric features is relatively coarse, and their physical fidelity remains relatively limited. For instance, only a few empirical models capture storm-driven density enhancements associated with Joule heating and neutral winds \cite{Brown2024UsingSolutions}. By contrast, more sophisticated physics-based models can represent these complexities more accurately but at a higher computational cost. Physics-based models (e.g., TIE-GCM, WAM-IPE) have also advanced significantly in recent years \cite{Brown2024UsingSolutions, Codrescu2012AModel, Bruinsma2023ThermosphereDrag}, although their computational intensity still limits widespread operational use \cite{Sutton2018AThermosphere,Mutschler2023AOperations,Elvidge2019UsingModelling,Mehta2018AModels}.

Operationally, models (both physics and empirical) are driven by nowcast or forecasted geomagnetic indices (e.g., F10.7, Ap, Dst), which can themselves become unreliable during extreme events\cite{Shprits2019NowcastingObservations, Yamazaki2024AssessmentIndices, Matzka2021TheActivity}. Storm events are the periods during which the largest errors in these models occur resulting in increased error in orbit propagation and degradation of the quality of key operational metrics such as probability of collision estimates \cite{Parker2024InfluencesAssessment}.

The flip side of the recent increase in the number of satellites in LEO has meant that GNSS data has emerged as a potentially globally scalable, high-resolution, near-real-time set of “signals of opportunity” for thermospheric density estimation. This opportunity is materialising through a rapid growth in the pool of available POD data from both scientific and commercial missions \cite{Arnold2023PreciseSatellites, Schreiner2022GFZProducts, Mutschler2023Physics-BasedData, Fitzpatrick2024ApplyingSatellite}. 
If the uncooperatively-derived densities that drive HASDM can achieve such strong performance, it stands to reason that assimilative models using a greater number of cooperative observations should be able to see even greater improvements. In view of the mounting criticality of storm-time density retrieval, the latest International Space Weather Action Teams Working Meeting \cite{Mehta2022SatelliteOperations} explicitly called for the community to explore new high-quality density data streams and ways of achieving “high-resolution monitoring of geomagnetic storms”. 

Despite the growing interest in POD-derived densities, little to no systematic observational characterization of their performance under geomagnetic storm conditions has been carried out. This gap in benchmarking limits the community’s ability to make informed decisions about how best to integrate POD-derived data into model development and operational workflows. Recent simulation-based work by \citeA{Ray2024ErrorMinimum} took an important step toward addressing this gap, concluding that a POD-accelerometry approach should outperform EDR-based approaches especially under lower drag scenarios (where drag acceleration $\lesssim 10^{-8}\,\mathrm{m\,s^{-2}}$). To date, however, no equivalent study exists using real observational data. This paper aims to fill that gap.

%%%% Justification/Scope for this work %%%%%
The results and analysis presented herein were developed with two overlapping communities in mind.
First and foremost are developers of next-generation assimilative models. The statistics presented support the characterization of observation-error variances and will aid in setting adaptive quality-control thresholds for POD-derived densities, ensuring that each observation is weighted appropriately inside the assimilative pipeline. The same metrics also provide a quantitative benchmark which will aid in choosing the most suitable inversion method for any POD stream at hand.
Secondly, agencies and researchers advocating greater transparency in data dissemination will gain a quantitative measure of the performance uplift that openly released POD streams may be able to deliver relative to the heritage datasets that currently drive most operational assimilative systems.

%Mini-Conclusions
We present the first storm-focused (n=43 storms) benchmark of POD-derived thermospheric densities, comparing the two main cooperative density inversion methods (Energy Dissipation Rate and POD-accelerometry) to two operational model outputs (JB2008 and NRLMSISE-00) and to one commonly used uncooperative density inversion method (TLE-based).

Using accelerometer-derived densities as a ground-truth, our results indicate that across the 1887 orbits studied, POD-accelerometry most accurately and precisely represents the behaviour of the thermosphere. We also find that EDR-based density inversion drops in performance as a function of the drag signal-to-noise ratio within the POD data. We compare the degree to which this degradation occurs and corroborate the predictions made by \citeA{Ray2024ErrorMinimum} in an observational context. We further highlight the degradation in precision (as well as accuracy) associated with lower drag signal-to-noise ratio. We conclude that POD-accelerometry should be the method of choice for storm-time data assimilation.

\section{Data and Methods}

\subsection{Spacecraft and Orbit Source selection}

\begin{table}
\centering
\caption{Candidate LEO spacecraft with publicly available SP3 precise-orbit data.
         Data-access URLs: GFZ Potsdam products are available from
         \url{https://isdc.gfz-potsdam.de};
         ESA Swarm data from \url{https://swarm-diss.eo.esa.int};
         Spire and PAZ orbits from the NASA CSDA Satellite Data Explorer
         (\url{https://csdap.earthdata.nasa.gov});
         COSMIC-2 from \url{https://data.cosmic.ucar.edu};
         Planet Labs data by agreement with Planet Labs, Inc.}
\label{tab:available-orbits}
\resizebox{\textwidth}{!}{
\begin{tabular}{llrrl}
\multicolumn{1}{l}{\textbf{Spacecraft}} &
\multicolumn{1}{l}{\textbf{Data provider}} &
\multicolumn{1}{r}{\textbf{$N_{\text{sat}}$}} &
\multicolumn{1}{r}{\textbf{Alt.\ (km)}} &
\multicolumn{1}{l}{\textbf{Latency}} \\
\toprule
GRACE-FO-A/B, TerraSAR-X/TanDEM-X & GFZ Potsdam   &   4 & $\sim$490       & 30\,min -- 2\,d \\
Swarm A/B/C                        & ESA Swarm      &   3 & $\sim$460       & $\sim$1\,d      \\
Planet Labs (Flock)                & Planet Labs    & 200+ & $\sim$475      & Unspecified      \\
Spire (LEMUR-2 constellation)      & NASA CSDAP     & 100+ & $\sim$400--600 & N/A              \\
PAZ (Spanish X-band SAR)           & NASA CSDAP     &   1 & $\sim$514       & N/A              \\
COSMIC-2                           & UCAR           &   6 & $\sim$520--720  & $\sim$1\,d       \\
Sentinel series                    & GFZ Potsdam    &   7 & $\sim$700       & 30\,min -- 2\,d \\
\bottomrule
\end{tabular}}
\end{table}

% \begin{table}
% \centering
% \caption{Selected Spacecraft for the Study}
% \label{tab:selected-missions}
% \begin{tabular}{|l|l|l|l|l|}
% \hline
% \textbf{Spacecraft} & \textbf{Launch Date} & \textbf{De-orbit Date} & \textbf{Altitude} & \textbf{Inclination} \\ \hline
% GRACE-FO-A          & 22 May 2018          & -                     & 480-490km               & 89°                  \\ \hline
% CHAMP               & 15 Jul 2000          & 19 Sep 2010           & 350-450km               & 87°                  \\ \hline
% \end{tabular}
% \end{table}

The sources of precise orbits that were considered are outlined in Table\ref{tab:available-orbits}. The Rapid Science Orbits (RSO) were selected for this study \cite{Schreiner2022GFZProducts}. The RSOs exhibit latencies of up to 2 days. Satellite Laser Ranging (SLR) residuals are approximately 1-2 cm for RSOs in LEO \cite{Schreiner2022GFZProducts}. According to \citeA{Selvan2023PreciseReview}, this level of accuracy places the RSOs within the top 20-35\% of POD literature over the past two decades. This accuracy level is considered above average yet achievable, ensuring applicability and replicability of the following results across various satellites.

\subsection{Selection of Storms}
Geomagnetic storms are notoriously poorly characterized by empirical density models 
\cite{Oliveira2021TheStorms, Oliveira2017ThermosphereEjections, Mutschler2023AOperations}, yet they represent periods of significant stress on satellite operations. Increasing the pool and quality of data pertaining to thermospheric behaviour during storm events is crucial for improving existing thermospheric models in two main areas \cite{Fang2022Space2022, Berger2023TheOperations}: as a source of data to be assimilated within nowcast models and as a source of scientific observations to enhance understanding of thermospheric behaviour.

The existing literature provides limited information on the performance of POD-based inversion during geomagnetic storms. Recognising the criticality of these periods for satellite operations, we identified storm-time as a key phase to evaluate the applicability and potential of POD-density inversion.

Different categories of geomagnetic storms (G1-G5) \cite{NOAASWPC2024NOAAScales} trigger distinct physical phenomena in the magnetosphere-ionosphere-thermosphere system and exhibit varying signatures in density fluctuations
\cite{Knipp2021TimelinesStorms, Astafyeva2017GlobalModeling, Laskar2023ThermosphericStorm, Oliveira2017ThermosphereEjections}. To evaluate our method across the spectrum of storm behaviours, we identified all periods corresponding to each storm category during the operational lifetime of each satellite. Table~\ref{tab:storm_distribution} summarises the resulting 49~storms (22 for CHAMP, 27 for GRACE-FO-A) across NOAA G-scale categories G2--G5, and Figure~\ref{fig:storm_timeline} places them in the context of the F10.7 solar-flux record.

The determination of the time window for storm analysis was guided by prior findings from \citeA{Oliveira2021TheStorms}, which noted that the thermosphere typically peaks in density approximately 12 hours following Sudden Storm Commencement (SSC) and reverts to baseline levels within about 72 hours. SSC is traditionally defined by a sharp increase in the horizontal magnetic field, usually associated with a southward turning of the IMF Bz component.

Our method diverged from that of \citeA{Oliveira2021TheStorms} in two key aspects. First, our analysis window spanned from 24 hours before to 32 hours after the storm's maximum Kp index. Although the reduced post-storm window did not always capture the complete return to calm conditions, it sufficed for our analysis needs. This time frame provided sufficient pre-storm data to quantify relative density increases while maintaining a manageable computational load. 

Secondly, our analysis centred on the time of the maximum Kp value rather than the southward turning of the Bz component. This decision was motivated by two considerations: the inherent noisiness of the Bz signal, which often requires manual annotation to identify the southward turn, and the robustness of the Kp index as a common indicator of geomagnetic activity across empirical atmospheric density models. Focusing on the Kp maximum allowed us to capture the principal phases of geomagnetic storms, providing a reliable framework for analysing density fluctuations across different storm categories and models.

It is important to note that the indices used in the models throughout this paper are the ``definitive" or ``post-processed" variants, not predicted. In practice, the discrepancy between forecasted and actual indices can contribute error on par with that of the models themselves \cite{Mutschler2023AOperations}. This discrepancy worsens during heightened solar activity: for instance, \citeA{Licata2020BenchmarkingDrivers} report that planetary a-index (``ap") forecasts are still highly uncertain even on the scale of a day or two, while predictions of Kp have been known to degrade significantly around solar maximum \cite{Shprits2019NowcastingObservations, Parker2024InfluencesAssessment}, which coincides with periods of heightened geomagnetic activity. Moreover, the three-hour cadence of Kp can mask rapid geophysical transitions, further complicating storm-time modelling \cite{Yamazaki2022GeomagneticHpo}. Such inaccuracies not only degrade thermospheric density estimates but also hamper satellite collision risk assessments \cite{Bussy-Virat2018EffectsObjects}. Further challenge arises from the current lack of a universal benchmarking framework for assessing predictive skill across different indices and forecast horizons \cite{Parker2024InfluencesAssessment, Liemohn2018ModelPredictions}. As such, the model results presented herein represent a best-case scenario for the models. The gap between GNSS-derived densities and real-time or forecast densities is likely to be larger.

\begin{table}[h]
\centering
\caption{Distribution of selected storms by NOAA G-scale category and satellite.}
\label{tab:storm_distribution}
\begin{tabular}{lcccccr}
\multicolumn{1}{l}{\textbf{Satellite}} &
\multicolumn{1}{c}{\textbf{G1}} &
\multicolumn{1}{c}{\textbf{G2}} &
\multicolumn{1}{c}{\textbf{G3}} &
\multicolumn{1}{c}{\textbf{G4}} &
\multicolumn{1}{c}{\textbf{G5}} &
\multicolumn{1}{r}{\textbf{Total}} \\
\toprule
CHAMP       & 0 &  0 &  9 & 9 & 4 & 22 \\
GRACE-FO-A  & 0 & 10 & 14 & 2 & 1 & 27 \\[2pt]
\cmidrule{6-7}
            &   &    &    &   &   & \textbf{49} \\
\bottomrule
\end{tabular}
\end{table}

\begin{figure}[H]
\centering
\noindent\includegraphics[width=\textwidth]{f107_storm_timeline.png}
\caption{Timeline of the 49 selected geomagnetic storms overlaid on the daily
F10.7 solar-flux record. Vertical lines mark each storm, coloured by NOAA
G-scale category. The CHAMP era (2001--2005) samples the declining phase of
Solar Cycle~23, while the GRACE-FO-A era (2019--2025) spans the ascending
phase of Solar Cycle~25.}
\label{fig:storm_timeline}
\end{figure}

\subsection{Density Retrieval Methods}

\subsubsection{Effective density}
\label{sec:effective-density}

Both density retrieval methods in this study produce \emph{effective
densities} in the sense of \citeA{Picone2005ThermosphericSets}: the
velocity-weighted average that, when held constant over a given arc,
reproduces the observed drag-induced energy loss. For a co-rotating atmosphere
(no winds) the definition reduces to

\begin{linenomath*}
\begin{equation}
\label{eq:rho_eff}
\rho_{\mathrm{eff}}
\;=\;
\frac{\displaystyle\int_{t_{0}}^{t_{1}} \rho\,v_{\mathrm{rel}}^{2}\,
  |\mathbf{v}|\,\mathrm{d}t}
     {\displaystyle\int_{t_{0}}^{t_{1}} v_{\mathrm{rel}}^{2}\,
  |\mathbf{v}|\,\mathrm{d}t}
\end{equation}
\end{linenomath*}

\noindent where $\mathbf{v}$ is the inertial velocity,
$\mathbf{v}_{\mathrm{rel}} = \mathbf{v} - \boldsymbol{\omega}_{\oplus}
\times\mathbf{r}$ is the velocity relative to the co-rotating atmosphere,
and $t_{0},\,t_{1}$ delimit the averaging arc. The weighting kernel
$v_{\mathrm{rel}}^{2}\,|\mathbf{v}|$ is proportional to the instantaneous
drag power; it emphasizes perigee—where the satellite is fastest and the
atmosphere densest—rather than treating all epochs equally as an arithmetic
mean would. This distinction is important for eccentric orbits and for any
comparison with quantities derived from integrated energy loss.

Although originally formulated between successive perigee passages
\cite{Picone2005ThermosphericSets}, Eq.~\eqref{eq:rho_eff} is valid for
\emph{any} choice of integration limits $t_{0}$ and $t_{1}$. When the arc
spans one full revolution (perigee to perigee) we refer to the result as
an \emph{orbit-effective} density; when a shorter arc is used we refer to it
as an \emph{arc-effective} density. In both cases the physical quantity is the
same Picone effective density of Eq.~\eqref{eq:rho_eff}—only the integration
interval differs.

Orbit-effective cadence (one value per revolution) is standard in storm-time
thermosphere studies \cite{Fitzpatrick2024ApplyingSatellite,
Chen2014Storm-timePropagation, Bruinsma2018SpaceOrbit,
Li2017ThermosphericMethod} and matches the resolution most commonly
assimilated by contemporary physics-based filters \cite{Matsuo2012DataDensity,
Sutton2018AThermosphere, Sutton2021TowardSatellites}. However, next-generation
data-assimilation systems are expected to ingest density at sub-orbital
cadences, and the feasibility of doing so depends on the drag signal
strength—which increases substantially during geomagnetic storms. We therefore
consider both orbit-effective and arc-effective resolutions in this study,
using Eq.~\eqref{eq:rho_eff} throughout.

The TU~Delft accelerometer-derived densities
\cite{Siemes2023NewGRACE-FO}, which serve as the reference truth in this
study, are provided as instantaneous values at 10-second cadence. To ensure
a like-for-like comparison, we compute effective densities from the
accelerometer time series by applying Eq.~\eqref{eq:rho_eff} over the same
arc used by the retrieval method under evaluation. This guarantees that all
quantities being compared represent the same physical average over the same
time interval.

The following density retrieval approaches are considered:
\begin{enumerate}
  \item \textbf{POD-Accelerometry} (Section~\ref{sec:pod-density-inversion}):
        Satellite accelerations are obtained by numerical differentiation of
        the precise orbit velocities. Conservative and major non-conservative
        forces are modelled and subtracted at each time step, yielding a
        pointwise drag-acceleration time series. Effective density is then
        obtained by integrating this series with the Picone weighting of
        Eq.~\eqref{eq:rho_eff} over the chosen arc.

  \item \textbf{Energy Dissipation Rate} (Section~\ref{sec:edr-method}):
        The change in orbital energy over each arc is computed from the
        ephemeris velocities and positions—without numerical differentiation.
        Gravitational and other non-drag forces are accounted for via work
        integrals along the arc. The residual energy change is equated to
        the work done by drag, yielding the Picone effective density by
        construction
        \cite{Sutton2021TowardSatellites, Fitzpatrick2024ApplyingSatellite}.

  \item \textbf{Accelerometer reference}:
        Orbit-effective and arc-effective accelerometer densities are computed
        from the native 10-second TU~Delft data
        \cite{Siemes2023NewGRACE-FO} using Eq.~\eqref{eq:rho_eff}, as
        described above. These data are widely recognized as a robust
        benchmark \cite{March2021, Mehta2017, Gondelach2020,
        Kuang2014MeasuringData, Emmert2015ThermosphericReview}.
\end{enumerate}

For each storm, the effective density time series from every retrieval method
is multiplicatively de-biased against the accelerometer reference using the
median log-ratio:
\begin{linenomath*}
\begin{equation}
\label{eq:debias}
\rho_{\text{method}}^{\ast}(t)
\;=\;
\rho_{\text{method}}(t)\;\exp\!\bigl(-\hat{\beta}\bigr),
\qquad
\hat{\beta}
\;=\;
\operatorname{median}\!\Bigl\{
  \ln\rho_{\text{method}}(t_{i}) - \ln\rho_{\text{ACC}}(t_{i})
\Bigr\}_{i}
\end{equation}
\end{linenomath*}

\noindent This multiplicative correction is appropriate because thermospheric
densities are approximately log-normally distributed
\cite{Emmert2015ThermosphericReview, Picone2005ThermosphericSets}; an additive
(arithmetic) offset would under-correct at high densities and over-correct at
low densities. The correction compensates for constant scaling errors arising
from uncertainties in the drag coefficient ($C_D$), the effective
cross-sectional area ($A$), or the spacecraft mass ($m$), which enter both
inversions as a fixed ballistic coefficient $B = C_D A / m$. In operational
settings such biases are typically absorbed by a fitted ballistic coefficient;
removing them here allows us to focus on relative storm-time variability.

To reduce clutter and facilitate interpretation, the shorthands in Table \ref{tab:method_acronyms} are used throughout the figures.

\begin{table}[h]
\centering
\caption{Acronyms and corresponding density-retrieval methods used in this study.}
\label{tab:method_acronyms}
\begin{tabular}{ll}
\multicolumn{1}{l}{\textbf{Acronym}} &
\multicolumn{1}{l}{\textbf{Method}} \\
\toprule
ACC   & Accelerometer-derived densities (TU~Delft) \\
POD-A & POD-accelerometry \\
EDR   & Energy Dissipation Rate \\
\bottomrule
\end{tabular}
\end{table}

\subsubsection{POD-accelerometry density inversion}
\label{sec:pod-density-inversion}

POD-accelerometry refers to a family of methods that recover atmospheric
drag from precise orbit ephemerides by isolating the drag acceleration
implicit in the tracking data. Two commonly used approaches are:
(i)~ingesting positions and velocities into a reduced-dynamic orbit
determination filter that simultaneously estimates density corrections
through a stochastic process, and (ii)~numerically differentiating the
velocities to yield accelerations directly, from which modelled non-drag
forces are subtracted to isolate the drag signal
\cite{Bezdek2010CalibrationAccelerations, Calabia2015ASignal,
Calabia2017ThermosphericOrbits}. \citeA{Ray2024ErrorMinimum} adopt
the first approach using a second-order Gauss--Markov process within an
iterated extended Kalman filter-smoother.

This study follows the numerical-differentiation strategy but departs from
earlier implementations in two important respects. First, whereas
\citeA{Calabia2015ASignal} and \citeA{Bezdek2010CalibrationAccelerations}
obtain accelerations by double differentiation of positions, we differentiate
the SP3 velocities only once. Because the velocities delivered by the POD
process are determined within the orbit fit rather than derived from
positions, a single differentiation amplifies high-frequency noise
substantially less than the double-differentiation approach. Second, rather
than applying temporal smoothing to the recovered density, we integrate the
raw pointwise drag-acceleration time series with the Picone velocity weighting
of Eq.~\eqref{eq:rho_eff} over a chosen arc. The integration inherently
suppresses stochastic noise while preserving arc-averaged fidelity, and
produces a quantity that is directly comparable to the accelerometer reference
densities computed with the same weighting
(Section~\ref{sec:effective-density}).

The SP3 ephemeris provides positions and velocities at 30-second intervals.
Each velocity component is interpolated to a 0.01\,s cadence using a cubic
spline (piecewise third-degree polynomial with not-a-knot boundary conditions;
knots at each 30\,s SP3 epoch). A Savitzky--Golay pre-filter (window of
21~points $= 0.21$\,s, polynomial order~7) is applied to the interpolated
velocities to suppress sub-second spline-interpolation artefacts before
differentiation. Accelerations are then obtained by central finite differences
at the 0.01\,s step size; interpolation finer than 0.01\,s was found to yield
negligible improvement. The resulting acceleration time series is downsampled
from 0.01\,s to 15\,s to reduce the computational burden of the subsequent
force-model evaluations.

The total observed acceleration $\mathbf{a}_{\text{obs}}$ is decomposed as
\begin{equation}
\mathbf{a}_{\text{obs}}
= \mathbf{a}_{\text{con}}
+ \mathbf{a}_{\text{drag}}
+ \mathbf{a}_{\text{srp}}
+ \mathbf{a}_{\text{erp}}
+ \mathbf{a}_{\text{other}}
\end{equation}
where $\mathbf{a}_{\text{con}}$ is the gravitational acceleration (including
third-body perturbations), $\mathbf{a}_{\text{srp}}$ is solar radiation
pressure, $\mathbf{a}_{\text{erp}}$ is Earth radiation pressure, and
$\mathbf{a}_{\text{other}}$ collects smaller effects such as thermal
re-radiation and antenna thrust. These residual terms are typically
$\lesssim$1\% of the drag acceleration, particularly during geomagnetic
storms when drag dominates
\cite{Bhattarai2022High-precisionSpacecraft, March2019CHAMPModelling,
Doornbos2010NeutralSatellites}, and their computation requires detailed
spacecraft thermal and attitude data that are not generally available. We
therefore set $\mathbf{a}_{\text{other}} = 0$.

The drag coefficient $C_D$ and cross-sectional area $A$ are held constant at
the values listed in Table~\ref{table:detailed_force_model}. In reality,
\hl{$C_D$ varies with thermospheric temperature and composition (principally the
O/N$_2$ ratio), with typical excursions of 5--15\% over a geomagnetic storm}
\cite{Pilinski2013Semi-empiricalObjects, Mehta2014ModelingSurfaces}. Holding
$C_D A$ constant introduces a proportional scaling error in the retrieved
density, which is largely absorbed by the multiplicative de-biasing of
Eq.~\eqref{eq:debias}. This simplification also ensures that the method
remains applicable to the broader catalogue of orbiting objects for which
$C_D$ and $A$ are poorly characterised.

\citeA{Bezdek2010CalibrationAccelerations, Ray2020GravitationalPrediction}
demonstrated that modelling the Earth's gravity field to high degree and order
is essential to prevent gravitational signal leakage into the drag residual.
We model the gravity field to degree and order 90
(Table~\ref{table:detailed_force_model}).

The drag acceleration is isolated as
\begin{equation}
\label{drag_eqn}
\mathbf{a}_{\text{drag}}
= \mathbf{a}_{\text{obs}}
- \mathbf{a}_{\text{con}}
- \mathbf{a}_{\text{srp}}
- \mathbf{a}_{\text{erp}}
\end{equation}
and projected onto the along-track direction defined by the
atmospheric-relative velocity
$\mathbf{v}_{\text{rel}} = \mathbf{v} -
(\boldsymbol{\omega}_{\oplus} \times \mathbf{r})$.
The instantaneous density at each timestep follows from the standard drag
equation:
\begin{equation}
\label{rho_eqn}
\rho(t_i) = \frac{2\,a_{\text{drag},\parallel}(t_i)\; m}
                  {C_D \, A \, v_{\text{rel}}^2(t_i)}
\end{equation}
where the symbols are defined as follows:

\begin{tabular}{r l}
$\mathbf{a}_{\text{obs}}$              : & Total observed acceleration (from numerical differentiation) \\
$\mathbf{a}_{\text{drag}}$            : & Acceleration due to drag (vector) \\
$a_{\text{drag},\parallel}$           : & Scalar along-track component of the drag acceleration \\
$\mathbf{v}_{\text{rel}}$             : & Satellite velocity relative to the co-rotating atmosphere \\
$v_{\text{rel}}$                       : & Magnitude of $\mathbf{v}_{\text{rel}}$ \\
$\mathbf{v}$                          : & Satellite velocity in inertial space \\
$\boldsymbol{\omega}_{\oplus}$        : & Angular rotation-rate vector of the Earth \\
$\mathbf{r}$                          : & Satellite position vector \\
$A$                                    : & Cross-sectional area of the satellite \\
$\rho$                                 : & Atmospheric density \\
$C_{D}$                                : & Drag coefficient \\
$m$                                    : & Satellite mass \\
\end{tabular}

\noindent The pointwise density series is then integrated over a chosen arc
using the Picone velocity weighting (Eq.~\ref{eq:rho_eff}) to yield the
arc-effective density. This integration step replaces the temporal smoothing
filters used in previous numerical-differentiation studies. Crucially, the
arc over which the density is extracted is independent of the span used for
the cubic-spline interpolation and differentiation, which always uses the
full multi-day SP3 ephemeris. The density-extraction arc is a sliding window
that can range from a few minutes to a full orbital revolution, and its
length controls the trade-off between temporal resolution and noise
suppression.

\begin{table}
\centering
\caption{Force modelling parameters and spacecraft characteristics used in this study.}
\resizebox{\textwidth}{!}{
\begin{tabular}{lll}
\multicolumn{1}{l}{\textbf{Parameter}} &
\multicolumn{1}{l}{\textbf{Value / Description}} &
\multicolumn{1}{l}{\textbf{Reference}} \\
\toprule
Third-body gravity    & Sun \& Moon point masses; DE421 ephemerides & \cite{Folkner2009The421, Montenbruck2000} \\
Gravity field         & EIGEN-6S4 $90\!\times\!90$                 & \cite{Forste2016EIGEN-6S4Toulouse} \\
Solid tides           & IERS 2014 conventions                      & \cite{Bizouard2019The2014} \\
Ocean tides           & FES 2004                                   & \cite{Lyard2006ModellingFES2004} \\
Relativistic effects  & Schwarzschild \& Lense--Thirring (Eq.~3.146) & \cite{Montenbruck2000} \\
Earth radiation       & Knocke $1\!\times\!1$\textdegree\ model    & \cite{Knocke1988EarthSatellites} \\
Solar radiation       & Cannonball ($C_r$); cone shadow model       & \cite{Montenbruck2000} \\
Aerodynamic drag      & Cannonball ($C_D$); density inverted-for    & \cite{Montenbruck2000} \\
Atmospheric winds     & Co-rotating atmosphere (no winds)           & \cite{Montenbruck2000} \\
Reference frame       & EME2000                                     & \cite{McCarthy1996IERSConventions} \\
Precession / nutation & IERS 2014 conventions                      & \cite{McCarthy1996IERSConventions} \\
Polar motion \& UT1   & IERS C04 14                                & \cite{Brzezinski2009SeasonalObservations} \\[4pt]
\multicolumn{1}{l}{\textbf{Spacecraft}} &
\multicolumn{1}{l}{\textbf{Mass\,(kg),\;Area\,(m$^2$),\;$C_D$,\;$C_r$}} &
\multicolumn{1}{l}{\textbf{Reference}} \\
\cmidrule{1-3}
GRACE-FO-A            & 600.2,\;1.04,\;3.2,\;1.5                  & \cite{Mehta2013} \\
CHAMP                 & 522.0,\;1.00,\;2.2,\;1.0                  & \cite{Mehta2017} \\
\bottomrule
\end{tabular}}
\label{table:detailed_force_model}
\end{table}

The method runs at 1.3 density estimates per second on a 10-core CPU laptop. For a 24 hour ephemeris sampled at 30-second intervals this results in a total run time of 36 minutes. Wall-clock time scales approximately inversely with the number of processing cores.

\subsubsection{Accelerometer-Derived Densities}

The latest version of the TU Delft accelerometer-derived atmospheric densities were obtained for CHAMP and GRACE-FO-A \cite{Siemes2023NewGRACE-FO}.

While accelerometer-derived densities are frequently regarded as a robust benchmark \cite{Sutton2021TowardSatellites, Mehta2022SatelliteOperations, Mutschler2023Physics-BasedData}, it is imperative to recognize that even these measurements are not devoid of errors. These errors persist despite the application of advanced correction techniques and high-fidelity models. For example, \citeA{Aruliah2019ComparingMeasurements} found that average zonal winds between 2001-2007 derived from the CHAMP satellite accelerometer were 1.5 to 2.0 times larger than those calculated from Doppler shifts observed by ground-based Fabry-Perot Interferometers at two high-latitude sites (auroral oval and polar cap). For this to be true, they demonstrated it would provide a challenge to the assumption,  held by all theoretical models, that the upper thermosphere has a high viscosity. One key argument against the size of the CHAMP winds is their similar magnitude to average plasma drifts observed by the EISCAT radar. Auroral zone plasma speeds are driven by strong electric fields generated by the solar wind magnetospheric dynamo. Through collisions between neutral particles and ions, the neutral winds can be accelerated, but there is an inertia, so the neutral winds rarely reach plasma speeds before the dynamic electric field changes. \citeA{Aruliah1996TheCycle} compared the seasonal and solar cycle average wind and plasma velocities at high latitudes, and  also monitored a common volume using 3 FPIs and 3 EISCAT radars \cite{Aruliah2004FirstRadar} which showed that auroral zone neutral winds are on average only 50\% of the magnitude of plasma velocities.

More recently, \citeA{Siemes2023NewGRACE-FO} also show a systematic offset between CHAMP crosswind and TIE-GCM model winds. The consequences on empirical and theoretical models of the global coverage and vast data output of satellite drag measurements compared with sparse ground-based independent measurements needs serious consideration for the future of modelling the LEO altitude regions. Additionally, \citeA{Doornbos2010NeutralSatellites} highlighted that the choice of wind models used in the density inversion process can alter the derived densities by up to 20\%. Thus, while accelerometer data serves as the best available benchmark in many studies, one should keep in mind its potential limitations.

\subsubsection{Energy Dissipation Rate Method}
\label{sec:edr-method}

The Energy Dissipation Rate (EDR) method recovers atmospheric density from
the change in orbital energy over an arc, avoiding the numerical
differentiation of velocity that is central to POD-accelerometry. Under
two-body dynamics the Jacobi energy of a satellite in the co-rotating
(ECEF) frame is conserved; departures from conservation are attributable to
non-gravitational forces, principally atmospheric drag
\cite{Picone2005ThermosphericSets, King-Hele1987TheLifetimes}. EDR-based
approaches underpin several operational density models
\cite{Storz2002HASDMRates, Bowman2003HighReview} and have been employed
extensively in recent literature
\cite{Hejduk2013ASolutions, Pilinski2016ImprovedSpecification,
Sutton2021TowardSatellites}.

This study follows the formulation of
\citeA{Sutton2021TowardSatellites} and \citeA{Ray2024ErrorMinimum}.
The Jacobi energy at each SP3 epoch is evaluated as
\begin{equation}
\xi = \frac{v_{\text{ECEF}}^2}{2}
    - \frac{\omega_{\oplus}^2(x^2 + y^2)}{2}
    - \frac{\mu}{r}
    - U_{\text{ns}}
\end{equation}
where $v_{\text{ECEF}}$ is the satellite speed in the ECEF frame,
$\omega_{\oplus}$ is the Earth's rotation rate, $\mu$ is the gravitational
parameter, $r$ is the geocentric distance, and $U_{\text{ns}}$ is the
non-spherical gravitational potential (EIGEN-6S4, degree and order 90).
Because the gravitational field is conservative, the potential energy is a
state function that depends only on position, not on the path taken. The
Jacobi energy at any epoch can therefore be evaluated directly from the
state vector at that epoch, and the energy change over an arc requires only
two evaluations of the gravity field (at the endpoints) rather than a
point-by-point summation along the entire trajectory. This makes the EDR
method substantially cheaper per arc than POD-accelerometry, an advantage
that becomes significant when scaling to the hundreds or thousands of
tracked objects in LEO envisaged for next-generation density assimilation.

The energy change over an arc $[t_0, t_1]$ attributable to drag is obtained
by subtracting the work done by the third-body (3BP) and radiation-pressure
(SRP, ERP) perturbations:
\begin{equation}
\Delta E_{\text{drag}} = -(\xi_1 - \xi_0)
  + \int_{t_0}^{t_1} \mathbf{a}_{\text{3BP}} \cdot \mathbf{v}_{\text{rot}}\, dt
  + \int_{t_0}^{t_1} \mathbf{a}_{\text{SRP}} \cdot \mathbf{v}_{\text{rot}}\, dt
\end{equation}
where $\mathbf{v}_{\text{rot}} = \mathbf{v} -
(\boldsymbol{\omega}_{\oplus} \times \mathbf{r})$ is the velocity in the
rotating frame. Following \citeA{Ray2024ErrorMinimum}, the work integrals for
SRP and ERP are included because their omission introduces appreciable error
at low-density regimes. These accelerations had already been computed at every
timestep for the POD-accelerometry inversion
(Section~\ref{sec:pod-density-inversion}), so their inclusion incurs no
additional cost.

The arc-effective density follows directly from equating the drag energy
dissipation to the work integral of the drag force
\cite{Sutton2021TowardSatellites}:
\begin{equation}
\rho_{\text{eff}} = \frac{2\,m\;\Delta E_{\text{drag}}}
                         {C_D\, A \;\displaystyle\int_{t_0}^{t_1}
                          v_{\text{rel}}^2\,|\mathbf{v}|\;dt}
\end{equation}
The denominator is precisely the Picone velocity-weighted integral of
Eq.~\eqref{eq:rho_eff}, so the EDR method produces the effective density by
construction. No post-processing smoothing or temporal averaging is required.
As with POD-accelerometry, the arc $[t_0, t_1]$ can range from a few minutes
to a full orbital revolution. Unlike POD-accelerometry, however, EDR uses the
SP3 positions and velocities directly at their native 30\,s cadence---there
is no spline interpolation or differentiation step. The arc endpoints alone
determine the gravitational energy change, with only the smaller third-body
and radiation-pressure work integrals requiring numerical quadrature along
the arc.

\subsection{Selection of Evaluation Metrics}

Thermospheric densities are approximately log-normally distributed
\cite{Emmert2015ThermosphericReview, Picone2005ThermosphericSets}; evaluation
metrics should therefore operate in log-space to treat multiplicative errors
symmetrically. Following \citeA{Fitzpatrick2024ApplyingSatellite}, we adopt
the standard deviation of the log-ratio as the primary scatter metric:
\begin{equation}
\sigma \;=\; \operatorname{std}\!\Bigl\{
  \ln\rho_{\text{method}}(t_i) - \ln\rho_{\text{ACC}}(t_i)
\Bigr\}_{i=1}^{N}
\end{equation}
which measures the dispersion of the retrieval around the reference after any
systematic bias has been removed. For practical interpretation, $\sigma$ is
converted to a percentage scatter
\cite{Fitzpatrick2024ApplyingSatellite}:
\begin{equation}
\text{SD\%} \;=\; 100\,\bigl(e^{\sigma} - 1\bigr)
\end{equation}
so that, for example, $\sigma = 0.10$ corresponds to
$\text{SD\%} \approx 10.5\%$. Lower values indicate tighter agreement with
the accelerometer reference.

In addition, we report the coefficient of determination ($r^{2}$) between
the retrieved and reference density time series, following
\citeA{Ray2024ErrorMinimum}:
\[
r^{2} \;=\; \biggl[\frac{
  \sum_{i}(\rho_{\text{method},i} - \bar{\rho}_{\text{method}})
         (\rho_{\text{ACC},i} - \bar{\rho}_{\text{ACC}})
}{
  \sqrt{\sum_{i}(\rho_{\text{method},i} - \bar{\rho}_{\text{method}})^{2}
        \;\sum_{i}(\rho_{\text{ACC},i} - \bar{\rho}_{\text{ACC}})^{2}}
}\biggr]^{2}
\]
While $\sigma$ quantifies the magnitude of relative errors, $r^{2}$ measures
how faithfully the retrieval reproduces the timing and relative amplitude of
density variations. Together, the two metrics provide a complete picture: a
method may have low scatter ($\sigma$) yet miss the temporal pattern
(low $r^{2}$), or vice versa.   

\section{Results and Discussion}

We evaluate both density retrieval methods across 48 geomagnetic storms
(26~CHAMP, 22~GRACE-FO-A) using accelerometer-derived densities as the
reference truth. All comparisons employ the log-normal metrics defined in
Section~\ref{sec:effective-density} and the Picone effective-density framework
of Eq.~\eqref{eq:rho_eff}. Results are organised around three findings:
(i)~both methods recover orbit-effective density with high fidelity and produce
statistically indistinguishable results; (ii)~increasing the fit-span
systematically improves precision; and (iii)~sub-orbital arcs of
30--45~minutes recover temporal resolution while maintaining useful accuracy.

\subsection{Orbit-Effective Validation}
\label{sec:orbit-effective-validation}

Figure~\ref{fig:scatter_heatmap} compares orbit-effective densities from
POD-A and EDR against accelerometer-derived effective densities computed over
matching time intervals. At one-orbit effective cadence ($\sim$90~min), both
methods agree closely with the accelerometer reference: POD-A achieves
$r^{2} = 0.980$ and $\text{SD\%} = 16.6\%$, while EDR achieves
$r^{2} = 0.988$ and $\text{SD\%} = 13.4\%$. The log-ratio bias
$\hat{\beta}$ is less than 0.005 for both methods, confirming that the
multiplicative de-biasing of Eq.~\eqref{eq:debias} effectively removes
systematic $C_D A$ scaling errors.

A direct comparison of the two methods against each other
yields $r^{2} = 0.977$ between their respective one-orbit effective densities
($n = 3{,}828$ orbits). When asking which method more closely matches the
accelerometer truth on any given orbit, EDR is closer only 48\% of the time.
The two methods are therefore statistically indistinguishable in accuracy at
orbit-effective timescales.
Given that EDR requires only two gravity-field evaluations per arc while
POD-A requires continuous force-model computation at every timestep, EDR is
the computationally preferred option when orbit-effective cadence is sufficient
(Section~\ref{sec:runtime}).

\begin{figure}[H]
\centering
\noindent\includegraphics[width=\textwidth]{density_scatter_heatmap.png}
\caption{Scatter heat-maps of retrieved density versus accelerometer-derived
density for POD-A (left) and EDR (right). Rows~1--3: orbit-effective densities
at 3-, 2-, and 1-orbit cadence compared against matched-interval accelerometer
effective densities. Row~4: 1-orbit effective density compared against
native-cadence (10\,s) accelerometer data. Row~5: sub-orbital densities
(fixed-optimal arc length) compared against native-cadence
accelerometer data. Annotations show $r^{2}$, $\sigma$, SD\%, $\beta$, and
sample size $n$. The dashed line denotes perfect agreement.}
\label{fig:scatter_heatmap}
\end{figure}

\subsection{Effect of Fit-Span on Precision}
\label{sec:fitspan}

Table~\ref{tab:fitspan} summarises performance across all five comparison
configurations shown in Figure~\ref{fig:scatter_heatmap}. Among the
orbit-effective comparisons (top three rows), all metrics improve
monotonically as the fit-span lengthens.

\begin{table}[h]
\centering
\caption{Log-normal evaluation metrics as a function of fit-span and
accelerometer reference cadence. ``Matched'' indicates that the accelerometer
truth was averaged over the same interval using Eq.~\eqref{eq:rho_eff};
``native'' indicates comparison against the 10\,s accelerometer time series.}
\label{tab:fitspan}
\resizebox{\textwidth}{!}{
\begin{tabular}{llcccccc}
&& \multicolumn{2}{c}{\textbf{SD\%}} &
     \multicolumn{2}{c}{$\mathbf{r^{2}}$} &
     \multicolumn{2}{c}{$\boldsymbol{\sigma}$} \\
\cmidrule(lr){3-4}\cmidrule(lr){5-6}\cmidrule(lr){7-8}
\multicolumn{1}{l}{\textbf{Fit-span}} &
\multicolumn{1}{l}{\textbf{ACC ref.}} &
\multicolumn{1}{c}{POD-A} & \multicolumn{1}{c}{EDR} &
\multicolumn{1}{c}{POD-A} & \multicolumn{1}{c}{EDR} &
\multicolumn{1}{c}{POD-A} & \multicolumn{1}{c}{EDR} \\
\toprule
3-orbit eff ($\sim$270\,min) & matched & 10.8 & \textbf{10.2} & 0.991 & \textbf{0.998} & 0.103 & \textbf{0.097} \\
2-orbit eff ($\sim$180\,min) & matched & 13.1 & \textbf{12.0} & 0.990 & \textbf{0.994} & 0.123 & \textbf{0.113} \\
1-orbit eff ($\sim$90\,min)  & matched & 16.6 & \textbf{13.4} & 0.980 & \textbf{0.988} & 0.154 & \textbf{0.126} \\[4pt]
1-orbit eff ($\sim$90\,min)  & native  & 58.2 & \textbf{56.7} & 0.798 & \textbf{0.805} & 0.459 & \textbf{0.449} \\[4pt]
Sub-orbital (best)           & native  & 42.7 & \textbf{39.8} & 0.924 & \textbf{0.941} & 0.355 & \textbf{0.335} \\
\bottomrule
\end{tabular}}
\end{table}

This improvement is expected: longer integration intervals average over more
of the stochastic noise in both the retrieval and the reference, and the
Picone weighting kernel of Eq.~\eqref{eq:rho_eff} ensures that the resulting
effective density remains a physically meaningful drag-weighted average rather
than a simple arithmetic mean.

\subsection{Sub-Orbital Resolution and the Precision--Resolution Tradeoff}
\label{sec:suborb-tradeoff}

While increasing the fit-span improves precision, it does so at the cost of
temporal resolution. Row~4 of Figure~\ref{fig:scatter_heatmap} illustrates
this directly: the same one-orbit effective densities from the previous
sections, when compared against native-cadence (10\,s) accelerometer data
rather than matched effective densities, yield $r^{2} = 0.80$ and
$\text{SD\%} \approx 58\%$. This degradation does not reflect a failure of
the retrieval method. Rather, it quantifies the density variability within a
single orbit that orbit-averaging cannot resolve. This distinction---between
method error and resolution-induced information loss---is central to
interpreting any orbit-effective density product.

Sub-orbital arcs provide a route to recovering temporal resolution while
retaining useful precision. To quantify this, each storm was processed at
17~arc lengths (2, 3, 5, 7, 10, 15, 20, 30, 45, 60, 75, 90, 100, 120,
135, 150 and 180~min) for both POD-A and EDR. For each
storm--satellite--method combination, the ``best'' arc length is the window
that maximises the Pearson correlation against the native-cadence (10\,s)
accelerometer truth. Using these per-storm best windows,
POD-A achieves $r^{2} = 0.924$ and $\text{SD\%} = 42.7\%$; EDR
achieves $r^{2} = 0.941$ and $\text{SD\%} = 39.8\%$
(Figure~\ref{fig:scatter_heatmap}, row~5). These
represent a substantial improvement over the orbit-effective-vs-native
comparison ($r^{2} = 0.80$), demonstrating that sub-orbital retrieval
recovers density variability that is otherwise lost to orbit averaging.

The optimal sub-orbital resolution is robustly 30--45~minutes across storms,
satellites, and methods. The per-storm optimal arc length peaks at 30~minutes
for CHAMP--POD-A and at 45~minutes for all other satellite--method
combinations, and the cross-storm spread in $\sigma$ tightens dramatically
between 20 and 45~minutes with only marginal improvement thereafter.

The accuracy--resolution trade-off curves
(Figure~\ref{fig:tradeoff_by_drag_regime}) reveal that performance at a given
arc length depends partly on drag signal strength: storms with stronger mean
drag acceleration (warmer colours) tend to achieve higher peak $r^{2}$ and
lower $\sigma$ across a wider range of arc lengths. However, the correlation
between drag acceleration and overall retrieval information content is modest
(Spearman $\rho \leq 0.28$, $p \geq 0.08$), indicating that storm-to-storm
variability is governed by factors beyond drag signal strength
alone---including storm morphology, orbit geometry, and data quality.

\begin{figure}[H]
\centering
\noindent\includegraphics[width=\textwidth]{tradeoff_drag_vs_regime.png}
\caption{Accuracy--resolution trade-off curves for all individual storms,
coloured by mean drag acceleration. Top: $\sigma$ vs.\ arc length.
Bottom: native-resolution $r^{2}$ vs.\ arc length. Black line: median.
High-drag storms (red) achieve lower $\sigma$ and higher $r^{2}$ at any
given arc length.}
\label{fig:tradeoff_by_drag_regime}
\end{figure}

\subsection{Computational Run-time}
\label{sec:runtime}
Run-time tests were performed under Python 3.11 on an Apple M1 (10-core)
processor equipped with 32\,GB of unified memory. For each retrieval
technique we bracketed the per-orbit inversion with the high-resolution
wall-clock timer \texttt{time.perf\_counter()} from Python's \texttt{time}
module, after all requisite input files (SP3 ephemerides, space-weather
indices, etc.) had been pre-loaded into memory. Across approximately
1\,000~orbits spanning both satellites, the mean execution time per orbit was
$0.185 \pm 0.003$\,s for POD-A and $0.031 \pm 0.001$\,s for EDR, giving a
roughly 6$\times$ speed advantage for EDR. I/O and plotting were excluded so
that the reported figures reflect only the computational cost of the retrieval
algorithms themselves. Tests were executed serially (single OS thread).

\section{Conclusions and Implications}

This study presents the first multi-timescale evaluation of POD-derived
thermospheric densities during geomagnetic storms, comparing POD-accelerometry
and Energy Dissipation Rate retrievals against accelerometer-derived effective
densities over 49~storms spanning CHAMP (22~storms) and GRACE-FO-A
(27~storms). Three principal findings emerge.

First, both methods recover orbit-effective density with high fidelity. At
one-orbit cadence, both achieve $r^{2} \geq 0.98$ and
$\text{SD\%} \leq 17\%$ relative to accelerometer-derived effective densities
computed over matching intervals; at three-orbit cadence, EDR reaches
$r^{2} = 0.998$ and $\text{SD\%} = 10.2\%$. The two methods produce
statistically indistinguishable orbit-effective densities
($r^{2} = 0.977$ between them; EDR closer to truth 48\% of the time). Since
EDR requires only endpoint gravity evaluations rather than continuous
force-model computation, it is computationally preferred at orbit-effective
cadence and for scaling to large populations of tracked objects.

Second, increasing the fit-span from one to three orbit-effective cadences
monotonically improves precision, reducing SD\% from 16.6\% to 10.8\%
(POD-A) and from 13.4\% to 10.2\% (EDR). This improvement reflects the
noise-suppression benefit of longer integration under the Picone
effective-density framework.

Third, longer fit-spans sacrifice temporal resolution. The same one-orbit
effective densities, when compared against native-cadence accelerometer data,
yield $r^{2} = 0.80$ and $\text{SD\%} \approx 58\%$---not because the
method fails, but because orbit-averaging cannot resolve sub-orbital density
variability. Sub-orbital retrieval at 30--45~minute arcs recovers this
temporal resolution ($r^{2} \approx 0.94$,
$\text{SD\%} \approx 40$--$43\%$), and this window is robustly optimal across
storms, satellites, and methods. These results provide practical guidance:
orbit-effective cadence for climatological studies and drag prediction,
sub-orbital cadence for storm-time dynamics where temporal resolution is
paramount.

The Picone effective-density framework adopted throughout this study resolves
a methodological ambiguity that has complicated previous evaluations of
POD-derived densities. By ensuring that the retrieval and the reference are
averaged over identical intervals with identical velocity weighting, the
comparison isolates genuine method error from the information loss inherent in
temporal averaging. Future benchmarks of POD-derived density products should
adopt this matched-truth protocol to avoid conflating the two sources of
discrepancy.

For developers of next-generation assimilative models, the statistics reported
here provide empirical support for observation-error characterisation. The
log-ratio standard deviation $\sigma$ at a given arc length and drag regime
can inform observation-operator tuning and quality-control thresholds,
ensuring that each ingested density observation is weighted
appropriately within the assimilation. The computational simplicity of EDR
makes it particularly attractive for operational systems that must process
hundreds or thousands of tracked objects in near-real time.

More broadly, the growing pool of cooperative POD data from scientific and
commercial missions offers a dataset of opportunity for thermospheric
monitoring \cite{Fitzpatrick2024ApplyingSatellite, Arnold2023PreciseSatellites}.
Even at the single-satellite level, both retrieval methods deliver
orbit-effective densities that are highly consistent with independent
accelerometer measurements. Harnessing cooperative data from a fraction of
today's $>$11,000~LEO satellites could supply a dense, low-latency
observational layer that informs thermospheric modelling precisely when
density errors are most operationally consequential: during geomagnetic
storms \cite{Parker2024SatelliteGannon, Parker2024InfluencesAssessment,
Berger2023TheOperations}.

\section{Future Work}
The present analysis draws exclusively on data from two near-polar, near-circular spacecraft orbits.
A logical next step is to extend this analysis to a broader range of orbital configurations.
Firstly in terms of orbit geometry: satellites spanning orbital geometries that more closely mirror contemporary mega-constellation configurations (in terms of inclinations, altitude, local-time precession rates, and spacecraft geometries).
Secondly in terms of data quality: if the use of POD data from commercial satellites constellations is to materialise, studies should be extended to use POD data that is of a comparable quality (around an order of magnitude worse than scientific quality POD data \cite{Arnold2023PreciseSatellites}). 
A non-exhaustive list of commercial and scientific candidate missions is provided in Table 1. Note that an important caveat in the extension of this work to more satellites will be that these platforms lack independent validation mechanisms such as accelerometers or laser retro-reflectors.

The current analyses contain relatively few samples below the $ 10^{-8}\,\mathrm{m\,s^{-2}}$ drag-acceleration regime where EDR performance deteriorates and variance grows. Extending the study to include a higher number of lower-drag arcs will fill this sparsely sampled corner of parameter space and clarify the drag-signal dependence of both EDR and POD-accelerometry approaches at higher altitudes or during more quiet times. 

%Text here ===>>>
%%
%  Numbered lines in equations:
%  To add line numbers to lines in equations,
%  \begin{linenomath*}
%  \begin{equation}
%  \end{equation}
%  \end{linenomath*}



%% Enter Figures and Tables near as possible to where they are first mentioned:
%
% DO NOT USE \psfrag or \subfigure commands.
%
% Figure captions go below the figure.
% Acronyms used in figure captions will be spelled out in the final, published version.

% Table titles go above tables;  other caption information
%  should be placed in last line of the table, using
% \multicolumn2l{$^a$ This is a table note.}
% NOTE that there is no difference between table caption and table heading in the final, published version
%
%----------------
% EXAMPLE FIGURES
%
% \begin{figure}
% \includegraphics{example.png}
% \caption{caption}
% \end{figure}
%
% Giving latex a width will help it to scale the figure properly. A simple trick is to use \textwidth. Try this if large figures run off the side of the page.
% \begin{figure}
% \noindent\includegraphics[width=\textwidth]{anothersample.png}
% \caption{caption}
% \label{pngfiguresample}
% \end{figure}
%
%
% If you get an error about an unknown bounding box, try specifying the width and height of the figure with the natwidth and natheight options. This is common when trying to add a PDF figure without pdflatex.
% \begin{figure}
% \noindent\includegraphics[natwidth=800px,natheight=600px]{samplefigure.pdf}
%\caption{caption}
%\label{pdffiguresample}
%\end{figure}
%
%
% PDFLatex does not seem to be able to process EPS figures. You may want to try the epstopdf package.
%

%
% ---------------
% EXAMPLE TABLE
%
% \begin{table}
% \caption{Time of the Transition Between Phase 1 and Phase 2$^{a}$}
% \centering
% \begin{tabular}{l c}
% \hline
%  Run  & Time (min)  \\
% \hline
%   $l1$  & 260   \\
%   $l2$  & 300   \\
%   $l3$  & 340   \\
%   $h1$  & 270   \\
%   $h2$  & 250   \\
%   $h3$  & 380   \\
%   $r1$  & 370   \\
%   $r2$  & 390   \\
% \hline
% \multicolumn{2}{l}{$^{a}$Footnote text here.}
% \end{tabular}
% \end{table}

%%%%%%%%%%%%%%%%%%%%%%%%%%%%%%%%%%%%%%%%%%%%%%%
% SIDEWAYS FIGURES and TABLES
% AGU prefers the use of {sidewaystable} over {landscapetable} as it causes fewer problems.
%
% \begin{sidewaysfigure}
% \includegraphics[width=20pc]{figsamp}
% \caption{caption here}
% \label{newfig}
% \end{sidewaysfigure}
%
%  \begin{sidewaystable}
%  \caption{Caption here}
% \label{tab:signif_gap_clos}
%  \begin{tabular}{ccc}
% one&two&three\\
% four&five&six
%  \end{tabular}
%  \end{sidewaystable}

%% If using numbered lines, please surround equations with \begin{linenomath*}...\end{linenomath*}
%\begin{linenomath*}
%\begin{equation}
%y|{f} \sim g(m, \sigma),
%\end{equation}
%\end{linenomath*}

%%% End of body of article

%%%%%%%%%%%%%%%%%%%%%%%%%%%%%%%%%%%%%%%%%%%%%%%
%% Optional Appendices go here
%
% The \appendix command resets counters and redefines section heads
%
% After typing \appendix
%
%\section{Here Is Appendix Title}
% will show
% A: Here Is Appendix Title
%
%\appendix
%\section{Here is a sample appendix}

%%%%%%%%%%%%%%%%%%%%%%%%%%%%%%%%%%%%%%%%%%%%%%%
% Optional Glossary, Notation or Acronym section goes here:
%
% Glossary is only allowed in Reviews of Geophysics
%  \begin{glossary}
%  \term{Term}
%   Term Definition here
%  \term{Term}
%   Term Definition here
%  \term{Term}
%   Term Definition here
%  \end{glossary}


%%%%%%%%%%%%%%%%%%%%%%%%%%%%%%%%%%%%%%%%%%%%%%%
% Acronyms
%% NOTE that acronyms in the final published version will be spelled out when used in figure captions.
%   \begin{acronyms}
%   \acro{Acronym}
%   Definition here
%   \acro{EMOS}
%   Ensemble model output statistics
%   \acro{ECMWF}
%   Centre for Medium-Range Weather Forecasts
%   \end{acronyms}


%%%%%%%%%%%%%%%%%%%%%%%%%%%%%%%%%%%%%%%%%%%%%%%
% Notation
%   \begin{notation}
%   \notation{$a+b$} Notation Definition here
%   \notation{$e=mc^2$}
%   Equation in German-born physicist Albert Einstein's theory of special
%  relativity that showed that the increased relativistic mass ($m$) of a
%  body comes from the energy of motion of the body—that is, its kinetic
%  energy ($E$)—divided by the speed of light squared ($c^2$).
%   \end{notation}




%%%%%%%%%%%%%%%%%%%%%%%%%%%%%%%%%%%%%%%%%%%%%%%
%
% DATA SECTION and ACKNOWLEDGMENTS
%
%%%%%%%%%%%%%%%%%%%%%%%%%%%%%%%%%%%%%%%%%%%%%%%

\section*{Open Research Section}
Accelerometer‐derived thermospheric densities for CHAMP and GRACE‐FO are openly available from the TU Delft database \cite{Siemes2023NewGRACE-FO} at \url{https://thermosphere.tudelft.nl/}. 
Precise orbit ephemerides for GRACE‐FO and CHAMP were obtained from the GFZ ISDC archive \cite{Schreiner2022GFZProducts} at \url{ftp://isdcftp.gfz-potsdam.de/}. 
TLEs are accessible via the public Space‐Track portal (\url{https://www.space-track.org/}). 
Geomagnetic and solar drivers were obtained from NOAA SWPC (\url{https://www.swpc.noaa.gov/})
and the Space Environment Technologies portal (\url{https://sol.spacenvironment.net/jb2008/}). 
NRLMSISE–00 densities were generated using the open‐source \texttt{pymsis} Python package \cite{Lucas2024PymsisModel}. 
JB2008 densities were computed using the Orekit \cite{Maisonobe2010OrekitApplications} (Python wrapper) JB2008 implementation, driven by the SOLFSMY and DTCFILE coefficient files obtained from Space Environment Technologies (\url{https://sol.spacenvironment.net/jb2008/}).

\acknowledgments
Enter acknowledgments here. This section is to acknowledge funding, thank colleagues, enter any secondary affiliations, and so on.

%%%%%%%%%%%%%%%%%%%%%%%%%%%%%%%%%%%%%%%%%%%%%%%
% REFERENCES and BIBLIOGRAPHY
%
% Added explicit bibliography style
\bibliographystyle{apacite}
\bibliography{references}
% don't specify bibliographystyle
%
%%%%%%%%%%%%%%%%%%%%%%%%%%%%%%%%%%%%%%%%%%%%%%%

%\bibliography{ enter your bibtex bibliography filename here }

%Reference citation instructions and examples:
%
% Please use ONLY \cite and \citeA for reference citations.
% \cite for parenthetical references
% ...as shown in recent studies (Simpson et al., 2019)
% \citeA for in-text citations
% ...Simpson et al. (2019) have shown...
%
%
%...as shown by \citeA{jskilby}.
%...as shown by \citeA{lewin76}, \citeA{carson86}, \citeA{bartoldy02}, and \citeA{rinaldi03}.
%...has been shown \cite{jskilbye}.
%...has been shown \cite{lewin76,carson86,bartoldy02,rinaldi03}.
%... \cite <i.e.>[]{lewin76,carson86,bartoldy02,rinaldi03}.
%...has been shown by \cite <e.g.,>[and others]{lewin76}.
%
% apacite uses < > for prenotes and [ ] for postnotes
% DO NOT use other cite commands (e.g., \citet, \citep, \citeyear, \nocite, \citealp, etc.).
%



\end{document}



More Information and Advice:

%%%%%%%%%%%%%%%%%%%%%%%%%%%%%%%%%%%%%%%%%%%%%%%
%
%  SECTION HEADS
%
%%%%%%%%%%%%%%%%%%%%%%%%%%%%%%%%%%%%%%%%%%%%%%%

% Capitalize the first letter of each word (except for
% prepositions, conjunctions, and articles that are
% three or fewer letters).

% AGU follows standard outline style; therefore, there cannot be a section 1 without
% a section 2, or a section 2.3.1 without a section 2.3.2.
% Please make sure your section numbers are balanced.
% ---------------
% Level 1 head
%
% Use the \section{} command to identify level 1 heads;
% type the appropriate head wording between the curly
% brackets, as shown below.
%
%An example:
%\section{Level 1 Head: Introduction}
%
% ---------------
% Level 2 head
%
% Use the \subsection{} command to identify level 2 heads.
%An example:
%\subsection{Level 2 Head}
%
% ---------------
% Level 3 head
%
% Use the \subsubsection{} command to identify level 3 heads
%An example:
%\subsubsection{Level 3 Head}
%
%---------------
% Level 4 head
%
% Use the \subsubsubsection{} command to identify level 3 heads
% An example:
%\subsubsubsection{Level 4 Head} An example.
%
%%%%%%%%%%%%%%%%%%%%%%%%%%%%%%%%%%%%%%%%%%%%%%%
%
%  IN-TEXT LISTS
%
%%%%%%%%%%%%%%%%%%%%%%%%%%%%%%%%%%%%%%%%%%%%%%%
%
% Do not use bulleted lists; enumerated lists are okay.
% \begin{enumerate}
% \item
% \item
% \item
% \end{enumerate}
%
%%%%%%%%%%%%%%%%%%%%%%%%%%%%%%%%%%%%%%%%%%%%%%%
%
%  EQUATIONS
%
%%%%%%%%%%%%%%%%%%%%%%%%%%%%%%%%%%%%%%%%%%%%%%%

% Single-line equations are centered.
% Equation arrays will appear left-aligned.

Math coded inside display math mode \[ ...\]
 will not be numbered, e.g.,:
 \[ x^2=y^2 + z^2\]

 Math coded inside \begin{equation} and \end{equation} will
 be automatically numbered, e.g.,:
 \begin{equation}
 x^2=y^2 + z^2
 \end{equation}


% To create multiline equations, use the
% \begin{eqnarray} and \end{eqnarray} environment
% as demonstrated below.
\begin{eqnarray}
  x_{1} & = & (x - x_{0}) \cos \Theta \nonumber \\
        && + (y - y_{0}) \sin \Theta  \nonumber \\
  y_{1} & = & -(x - x_{0}) \sin \Theta \nonumber \\
        && + (y - y_{0}) \cos \Theta.
\end{eqnarray}

%If you don't want an equation number, use the star form:
%\begin{eqnarray*}...\end{eqnarray*}

% Break each line at a sign of operation
% (+, -, etc.) if possible, with the sign of operation
% on the new line.

% Indent second and subsequent lines to align with
% the first character following the equal sign on the
% first line.

% Use an \hspace{} command to insert horizontal space
% into your equation if necessary. Place an appropriate
% unit of measure between the curly braces, e.g.
% \hspace{1in}; you may have to experiment to achieve
% the correct amount of space.


%%%%%%%%%%%%%%%%%%%%%%%%%%%%%%%%%%%%%%%%%%%%%%%
%
%  EQUATION NUMBERING: COUNTER
%
%%%%%%%%%%%%%%%%%%%%%%%%%%%%%%%%%%%%%%%%%%%%%%%

% You may change equation numbering by resetting
% the equation counter or by explicitly numbering
% an equation.

% To explicitly number an equation, type \eqnum{}
% (with the desired number between the brackets)
% after the \begin{equation} or \begin{eqnarray}
% command.  The \eqnum{} command will affect only
% the equation it appears with; LaTeX will number
% any equations appearing later in the manuscript
% according to the equation counter.
%

% If you have a multiline equation that needs only
% one equation number, use a \nonumber command in
% front of the double backslashes (\\) as shown in
% the multiline equation above.

% If you are using line numbers, remember to surround
% equations with \begin{linenomath*}...\end{linenomath*}

%  To add line numbers to lines in equations:
%  \begin{linenomath*}
%  \begin{equation}
%  \end{equation}
%  \end{linenomath*}



